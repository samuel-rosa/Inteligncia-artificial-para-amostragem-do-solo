Como dissemos acima, a configuração das amostras para o mapeamento pedométrico
precisa atender à diversos objetivos. No spsann, esse problema de otimização
multiobjetivo é formulado da seguinte forma:

\begin{equation}
\label{eqn:multiobjetivo}
\mathbf{f}(\mathbf{x}) = (f_1(\mathbf{x}), f_2(\mathbf{x}), 
                          \ldots, f_k(\mathbf{x}))
\end{equation}

onde $\mathbf{f}(\mathbf{x})$ é um vetor onde cada elemento é uma função
objetivo, desde 1 até \textit{k}. Como cada função objetiva resulta um valor, 
os elementos do vetor $\mathbf{f}(\mathbf{x})$ precisam ser agregados para 
formar uma única medida e, assim, resolvermos o nosso problema de otimização
 multiobjetivo. A agregação no spsann é feita usando o método da soma ponderada:

\begin{equation}
\label{eqn:soma}
U = \sum_{i=1}^{k} w_if_i(\mathbf{x})
\end{equation}

onde $U$ é a chamada função utilidade, o resultado da soma dos resultados 
individuais de cada uma das $k$ funções objetivas multiplicadas pelos seus 
respectivos pesos $w$. Os pesos servem para o usuário articular suas possíveis
 preferências em relação à uma ou outra função objetiva. Em geral, a definição 
dos pesos é fundamentalmente subjetiva, dependendo da experiência do usuário. O
 spsann não fornece qualquer recomendação sobre a escolha dos pesos, exceto 
pela exigência de que sejam positivos e maiores do que zero, e que sua soma 
seja igual a 1. Caso não haja qualquer preferência, usa-se o mesmo peso para 
todas as funções objetivas.

A soma ponderada é o método mais usado para agregar os valores das funções 
objetivas em uma função utilidade. Entretanto, os valores de energia das funções
 objetivas costumam ter intervalos e/ou ordens de magnitude diferentes. Quando 
uma função objetiva produz valores de energia muito altos relativamente às 
outras funções objetivas, diz-se que ela possui uma dominância numérica, o que
 lhe dá um maior peso intrínseco na hora de calcular a função utilidade. Na 
prática, isso significa que a otimização está “mais preocupada” em atender 
àquele objetivo do que aos outros. Isso acontece, por exemplo, com o algoritmo
 de Minasny & McBratney (2006), usado para a otimização da configuração amostral
 para estimativa do componente determinístico. Naquele, a primeira função 
objetiva (\textit{O1}) possui dominância numérica sobre as outras duas 
(\textit{O2} e \textit{O3}). Isso porque os valores de energia de \textit{O1} 
são computados usando números inteiros (número de pontos por estrato amostral 
marginal). Já os valores de energia de O2 e O3 são computados usando números 
decimais entre 0 e 1 (proporção de pontos em cada classe amostral marginal), e 
-1 e 1 (coeficiente de correlação linear), respectivamente.

\begin{equation}
\label{eqn:escalonamento}
f^{es}_i = \frac{f_i(\mathbf{x}) - f^\circ_i}{f^{max}_i - f^\circ_i}
\end{equation}
