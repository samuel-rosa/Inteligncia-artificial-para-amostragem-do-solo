O mapeamento pedométrico, ou mapeamento do solo usando modelos 
estatísticos, explora a correlação entre as propriedades do solo
(variáveis) e as características do ambiente (covariáveis) em um 
número finito de pontos (observações). Diferente do mapeamento 
tradicional, onde o pedólogo vai para o campo já com um modelo de 
variação espacial do solo, e atualiza-o a cada nova observação, no
mapeamento pedométrico o modelo de variação espacial do solo 
geralmente só é conhecido depois que todas as observações tenham sido
feitas. Assim, a configuração espacial das observações precisa ser 
ótima para identificar e estimar ambos os componentes (1) determinístico
e (2) estocástico de variação espacial do solo, e permitir realizar 
(3) predições espacial com a menor incerteza possível. Fazer uma 
amostragem para atender aos três objetivos simultaneamente é um grande
desafio, dado que para cada objetivo são necessárias configurações
amostrais diferentes.

Existem muitos algoritmos e programas de computador (privados e 
públicos) para otimizar a configuração de amostras para o mapeamento
pedométrico. Nenhum deles visa os três objetivos citados acima quando
somos completamente ignorantes sobre o modelo de variação espacial 
do solo. Muitos pesquisadores usam métodos tradicionais (sub-ótimos) 
para obter observações para a calibração de modelos estatísticos para
o mapeamento do solo. Alguns algoritmos estão disponíveis apenas em 
artigos e livros didáticos. Talvez essa dispersão e (relativamente) 
pequena disponibilidade estejam dificultando o seu maior uso e a 
continuação do seu desenvolvimento.

Nesse trabalho apresentamos \textbf{spsann}, um novo pacote do R para a
otimização da configuração amostral para o mapeamento pedométrico. 
O R foi escolhido por ser o ambiente mais popular de processamento
e análise de dados. Além disso, o R é um software livre e de código
aberto, e relativamente fácil se comparado a outros ambientes 
computacionais. A estratégia escolhida para construir o \textbf{spsann}
foi iniciar a partir de pacotes já existentes, como
\textbf{intamapInteractive} (Edzer Pebesma, Jon Skoien, et al.) e 
\textbf{clhs} (Pierre Roudier). Entre as principais ferramentas usadas 
para construir e manter o spsann estão Rstudio, \textbf{roxygen2},
\textbf{Rcpp}, e GitHub.

O projeto de desenvolvimento do spsann possui cinco componentes 
básicos: a) as funções objetivas, b) a formulação do problema de
otimização multiobjetivo, c) o recozimento simulado espacial, d) as
estratégias de ganho em eficiência, e e) as ferramentas de 
visualização.

As funções objetivas... 
O spsann possui cinco funções objetivas singulares implementadas:

\begin{description}
\item [\verb|optimCORR|] associação/correlação entre as covariáveis
      (componente determinístico)
\item [\verb|optimDIST|] distribuição marginal das covariáveis 
      (componente determinístico)
\item [\verb|optimMSSD|] média quadrática da distância mais próxima
      (predição espacial – krigagem)
\item [\verb|optimMKV|] variância da krigagem média/máxima (predição
      espacial – krigagem)
\item [\verb|optimPPL|] número de pontos/pares de pontos por lag 
      (componente estocástico)
\end{description}

